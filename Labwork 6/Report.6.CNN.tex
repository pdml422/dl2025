\documentclass{article}
\usepackage{graphicx} % Required for inserting images

\title{Labwork 6: CNN}
\author{Phi Doan Minh Luong - 2440046}
\date{May 2025}

\begin{document}

\maketitle

\setlength\parindent{0pt}

\section{Design}
\subsection{Convolutional Layers}
- The architecture follows the VGG19 design, consisting of convolutional layers with increasing filter sizes (64, 128, 256, 512)

- Each convolutional layer uses a kernel size of 3x3, stride of 1, and padding of 1 to preserve spatial dimensions

- Batch normalization is applied after each convolution to stabilize training

- ReLU activation is used for non-linearity

\subsection{Pooling Layers}
- Max-pooling layers with a kernel size of 2x2 and stride of 2 are interspersed to reduce spatial dimensions

\subsection{Fully Connected Layers}
- After the convolutional layers, the feature maps are flattened and passed through three fully connected layers

- The first two fully connected layers have 4096 units each, followed by ReLU activation and dropout for regularization

- The final fully connected layer outputs predictions for the number of classes

\subsection{Input and Output}
- The network accepts input images of size 224x224x3 and outputs class probabilities for the specified number of classes

\section{Implementation}
\subsection{Class Definition}

- The VGG19 class is implemented using PyTorch's nn.Module

- The convolutional layers are dynamically created based on the VGG\_architecture list

\subsection{Forward pass}
- The input is passed through the convolutional layers, flattened, and then processed by the fully connected layers

\subsection{Training and Testing}
- The network is trained using the CIFAR-10 dataset with cross-entropy loss and the Adam optimizer

- Accuracy is evaluated on the test set

\end{document}
